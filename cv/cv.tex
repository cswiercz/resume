\documentclass{article}
\usepackage{fullpage}
\usepackage{enumitem}
\usepackage{parskip}
\usepackage{hyperref}

%\setlist{nolistsep}
\setlength{\parskip}{2pt}
\pagestyle{empty}

\usepackage[margin=2.4cm]{geometry}


%%%%%%%%%%%%%%%%%%%%%%%%%%%%%%%%%%%%%%%%%%%%%%%%%%%%%%%%%%%%%%%%%%%%%%%%%%%%%%%
\begin{document}
%%%%%%%%%%%%%%%%%%%%%%%%%%%%%%%%%%%%%%%%%%%%%%%%%%%%%%%%%%%%%%%%%%%%%%%%%%%%%%%

\section*{Christopher J. Swierczewski \hspace{4.9cm} Curriculum Vitae}

\begin{tabular}{lrl}
  Department of Applied Mathematics \hspace{5cm} & Cell: & 253.223.3721 \\
  University of Washington                       & Fax:     & 206.685.1440 \\
  Lewis Hall \#202, Box 353925        & Website: & \tt www.cswiercz.info \\
  Seattle, WA 98195                   & E-mail:  & \tt cswiercz@gmail.com
\end{tabular}




%%%%%%%%%%%%%%%%%%%%%%%%%%%%
\section*{Areas of Interest}
%%%%%%%%%%%%%%%%%%%%%%%%%%%%

\begin{tabular}{ll}
  \bf General: & Complex Algebraic Geometry, Partial Differential Equations \\
         & Numerical Analysis, Computational Mathematics \\
  \bf Emphasis: & Riemann Surfaces, Computational Geometry, Abelian Functions\\
                & Symbolic and Numerical Computation, Nonlinear Waves
\end{tabular}



%%%%%%%%%%%%%%%%%%%%
\section*{Education}
%%%%%%%%%%%%%%%%%%%%



\begin{itemize}
  \item {\it Ph.D. in Applied Mathematics}, University of Washington,
    Seattle, Expected March 2016 \\ Advisor: Bernard Deconinck
  \item {\it M.S. in Applied Mathematics}, University of Washington,
    Seattle, June 2010 \\ Masters Project: {\it A Python Implementation
      of Chebyshev Functions}
  \item {\it B.S. in Mathematics (Comprehensive) with Distinction},
    University of Washington, Seattle, June 2008 \\ Thesis: {\it
      Connections Between the Sato-Tate Conjecture and the Generalized
      Riemann Hypothesis} \\ Advisor: William Stein
\end{itemize}


%%%%%%%%%%%%%%%%%%%%
\section*{Research Projects}
%%%%%%%%%%%%%%%%%%%%



\begin{itemize}
  \item {\it {\sc abelfunctions}: A Python library for computing with
    complex algebraic curves, Riemann surfaces, and Abelian functions.}
    \url{http://abelfunctions.cswiercz.info}
    \begin{itemize}
    \item Design and implementation of algebraic-numerical hybrid
        tools for computing with Abelian functions and Riemann surfaces
        in a Python-based open-source mathematical software package,
        ``abelfunctions''.
      \item Applying research results to computing periodic solutions to a large
        class of nonlinear partial differential equations using techniques from
        computational geometry, numerical analysis, and algebraic geometry.
      \item Focus on designing high performance code in both computer algebraic
        and numerical aspects of the software package in a Cython / C back-end
        with an easy to use Python front-end.
      \item Advised two undergraduate students in related projects on
        quickly and accurately computing Riemann theta functions.
      \item Open-source code available on GitHub:
        \url{https://github.com/cswiercz/abelfunctions}
    \end{itemize}
  \item {\it {\sc zipper} Development}
    \begin{itemize}
      \item {\sc zipper} is a collection of Fortran programs developed by Donald
        Marshall of the Department of Mathematics at the University of
        Washington for computing conformal maps.
      \item Integrated the software into Sage and added a web-based, interactive
        front-end.
      \item Added functionality to the core library including routines for
        computing the Carleson map.
    \end{itemize}
  \item {\it Masters Project: A Python Implementation of Chebyshev
    Functions}
    \begin{itemize}
      \item Studied the Chebfun system developed by Lloyd Trefethen et. al. and
        implemented core functionality in Python using the Numpy/Scipy Python
        libraries.
      \item Collaborated with Trefethen on porting Chebfun to an open-source
        license.
    \end{itemize}
  \item {\it {\sc clawpack} Development}
    \begin{itemize}
      \item {\sc clawpack} is a Fortran program developed by Randall Leveque for
        numerical solutions to hyperbolic partial differential equations.
      \item Performed foundational work on conversion of CLAWPACK to a dynamic
        library.
      \item Attended Scipy 2009 conference on scientific computing in Python.
    \end{itemize}
  \item {\it Senior Thesis: Connections Between the Sato-Tate Conjecture
    and the Generalized Riemann Hypothesis}
    \begin{itemize}
      \item Proved equivalence of Sato-Tate conjecture and Generalized Riemann
        Hypothesis for elliptic curves over the rational numbers.
      \item Performed computational verification of the Sato-Tate conjecture for
        rational elliptic curves. Results published in a paper by Barry Mazur in
        the AMS Bulletin v.45 no.2.
    \end{itemize}

\end{itemize}



%%%%%%%%%%%%%%%%%%%%%%%%%%%%
\section*{Professional Experience}
%%%%%%%%%%%%%%%%%%%%%%%%%%%%



\begin{itemize}
  \item {\it Research Mathematician}, Institute for Defense Analysis:
    Center for Communications Research, La Jolla, CA. June – August 2012
  \item {\it Software Developer}, Simulab Corporation, Seattle,
    WA. January 2009 -- March 2009.
    \begin{itemize}
      \item Researched theory and applications of Hidden Markov Models
        to problems in control theory.
      \item Implemented Hidden Markov Model C/C++ library, GHMM, in
        the EDGE project: a surgical trainer for evaluating surgeon
        performance.
    \end{itemize}
  \item {\it Sage: Mathematics Software Developer}, Department of
    Mathematics, University of Washington, Seattle, WA. September 2007
    -- September 2008.
    \begin{itemize}
      \item Implemented the Opentick financial data acquisition
        API. Created a new mathematical finance package. Devised
        methods of wrapping asynchronous functions in a synchronous
        environment.
      \item Designed tests and wrote documentation for advanced
        mathematical functions in Python, Cython, and C/C++ under a
        UNIX environment.
      \item Collaborated with other Sage developers from Germany,
        France, and Canada.
    \end{itemize}
  \item {\it Applied Research Mathematician}, National Security Agency,
    Ft. Meade, MD. June -- August 2007.
    \begin{itemize}
      \item Applied algebraic, probabilistic, and statistical methods
        to improve cryptanalytic attacks against telecommunication
        encryption standards.
      \item Collaborated with mathematicians in researching
        cryptographic algorithm weaknesses. Implemented algorithms in
        C.
      \item Received background check in Spring 2007 and TOP SECRET
        clearance.
    \end{itemize}
  \item {\it Teaching Assistant and Math Camp Counselor}, Department of
    Mathematics, University of Washington, Seattle, WA. June -- August
    2005 and 2006.
\end{itemize}


%%%%%%%%%%%%%%%%%%%%
\section*{Publications}
%%%%%%%%%%%%%%%%%%%%

\begin{itemize}
\item B. Deconinck, M. S. Patterson, C. Swierczewski, {\it Computing the
  Riemann Constant Vector}, {\it Submitted for publication}, 2015,
  \url{http://www.cswiercz.info/assets/files/rcv.pdf}.
\item C. Swierczewski, {\it Introduction to Differential Equations Using
  Sage (Book Review)}, SIAM Review, Book Reviews, 56(2), 373--382.
  \url{http://dx.doi.org/10.1137/140973669}.
\item C. Swierczewski, B. Deconinck, {\it Computing Riemann theta
  functions in Sage with applications}, Mathematics and Computers in
  Simulation, Available online 16 May 2013, ISSN 0378-4754,
  \\ \url{http://dx.doi.org/10.1016/j.matcom.2013.04.018}.
\end{itemize}

%%%%%%%%%%%%%%%%%%%%
\section*{Professional Activities and Service}
%%%%%%%%%%%%%%%%%%%%

\subsection*{Session Organizer / Co-Organizer}
\begin{itemize}
  \item {\it AMS Special Session on Nonlinear Waves and Coherent Structures},
    2016 Joint Mathematics Meetings, American Mathematical Society, Seattle,
    WA. 6-9 January 2016.
  \item {\it Special Session on Riemann Theta Functions}, 1st SIAM-SIAG on
    Applied Algebraic Geometry, Society for Industrial and Applied Mathematics
    Conference, Raleigh, NC. 6-9 October, 2011.
\end{itemize}

\subsection*{Invited Speaker}
\begin{itemize}
  \item {\it Calculus on Riemann Surfaces in Python}, Symbolic
    Computation Seminar, North Carolina State University, Rayleigh,
    North Carolina. 18-20 March 2013.
\end{itemize}

\subsection*{Conferences and Workshops}
\begin{itemize}
\item {\it Computing Solutions to the Kadomtsev-Petviashvili Equation}, 2016
  Joint Mathematics Meetings, Americano Mathematical Society, Seattle,
  Washington. 6-9 January 2016.
  \item {\it Calculus on Riemann Surfaces in Python}, The Eighth
    Annual IMACS Conference on Nonlinear Evolution Equations and Wave
    Phenomena, Athens, Georgia. 25-28 March 2013.
  \item {\it Some Computational Problems Using Riemann Theta
    Functions in Sage}, AMS 2011 Fall Western Section Meeting, Salt
    Lake City, Utah. 22-23 October 2011.
  \item {\it Some Computational Problems Using Riemann Theta
    Functions in Sage}, SIAM Conference on Applied Algebraic
    Geometry, Chapel Hill, North Carolina. 6-9 October 2011.
  \item {\it A Python Implementation of Chebyshev Functions},
    International Council for Industrial and Applied Mathematics
    (ICIAM), Vancouver, British Columbia, Canada. 18-20 July 2011.
  \item {\it Computing Bitangents of Quartics Using Riemann Theta
    Functions} (Poster), Algebraic Geometry in the Sciences, Center
    for Mathematics and Applications, Oslo, Norway. 10-14 January
    2011.
  \item {\it Computing Bitangents of Quartics Using Riemann Theta
    Functions} (Poster), The Higher Genus Sigma Function and
    Applications, International Center for Mathematical Sciences,
    Edinburgh, UK. 11-15 October 2011.
\end{itemize}

\subsection*{Seminars and Colloquia}
\begin{itemize}
  \item {\it Object-Oriented Design in Scientific Software (Part 2)},
    Numerical Analysis Research Group, Seattle, Washington. 24 April
    2014.
  \item {\it Object-Oriented Design in Scientific Software (Part 1)},
    Numerical Analysis Research Group, Seattle, Washington. 17 April
    2014.
  \item {\it An Introduction to GPGPU Computing (Part 2)}, Applied
    Mathematics Special Topics Seminar, Seattle, Washington. 15
    November 2012.
  \item {\it An Introduction to GPGPU Computing (Part 1)}, Applied
    Mathematics Special Topics Seminar, Seattle, Washington. 8
    November 2012.
  \item {\it A Sample of Scientific Computing in Python},
    Undergraduate Mathematical Sciences Seminar, Seattle,
    Washington. 17 May 2012.
  \item {\it Abelfunctions: Software for Computing with Riemann
    Surfaces}, Mathematical Methods Seminar, Seattle, Washington. 27
    March 2012.
  \item {\it Determinantal Representations of Algebraic Curves and
    Riemann Theta Functions}, Convex Algebraic Geometry Seminar,
    Seattle, Washington. 18 February 2011.
  \item {\it Polynomial Approximations to Functions}, Undergraduate
    Mathematical Sciences Seminar, Seattle, Washington. 19 January 2011.
  \item {\it Computing Two--Phase Solutions to the
    Kadomtsev--Petviashvili Equation}, Mathematical Methods Seminar,
    Seattle, Washington. 4 January 2011.
  \item {\it Computing Three--Phase Solutions to the
    Kadomtsev--Petviashvili Equation}, Solitons and Nonlinear Waves
    Course: Final Talks, Seattle, Washington. 9 December 2010.
\end{itemize}


\subsection*{Service}
\begin{itemize}
  \item {SIAM University of Washington Student Chapter: President},
    University of Washington, (September 2013 -- August 2014)
  \item {SIAM University of Washington Student Chapter: Math Fair
    Co-Organizer}, Lockwood Elementary School, Seattle, WA. (December
    2011, 2012)
  \item {Math Hour Olympiad: Judge}, University of Washington, (June
    2013, 2014, 2015)
  \item {SIAM University of Washington Student Chapter: Webmaster},
    University of Washington, (September 2011 -- August 2013)
  \item {Applied Mathematics Systems Administrator}, University of
    Washington, (September 2011 -- March 2014)
  \item {Numerical Analysis Research Club Moderator}, University of
    Washington, Department of Applied Mathematics. (Winter -- Spring
    2009)
\end{itemize}



%%%%%%%%%%%%%%%%%
\section*{Awards}
%%%%%%%%%%%%%%%%%

\begin{itemize}
  \item SIAM Student Chapter Certificate of Recognition, Society for
    Industrial and Applied Mathematics, 2014.
  \item Boeing Service Award, University of Washington, Applied
    Mathematics, 2013.
  \item American Mathetmatical Society Sectional Meeting Travel Grant,
    October 2011.
  \item University of Alaska Fairbanks Travel Grant, January 2011.
\end{itemize}


%%%%%%%%%%%%%%%%%%%%%%%%%%%%%%
\section*{Teaching Experience}
%%%%%%%%%%%%%%%%%%%%%%%%%%%%%%



\begin{itemize}
  \item Instructor, University of Washington, Seattle:

    \begin{tabular}{ p{2.25cm} p{10.5cm} p{2.25cm} }
      - AMATH 301: & Beginning Scientific Computing      & Summer 2014 \\
      - AMATH 301: & Beginning Scientific Computing      & Summer 2011 \\
    \end{tabular}

  \item Teaching Assistant, University of Washington, Seattle:

    \begin{tabular}{ p{2.25cm} p{10.5cm} p{2.25cm} }
      - AMATH 301: & Beginning Scientific Computing      & Winter 2015 \\
      - AMATH 351: & Introduction to Differential Equations and Applications
                   & Autumn 2014 \\
      - AMATH 301: & Beginning Scientific Computing      & Winter 2011 \\
      - AMATH 301: & Beginning Scientific Computing      & Autumn 2010 \\
      - MATH 125:  & Calculus with Analytic Geometry II  & Spring 2010 \\
      - MATH 124:  & Calculus with Analytic Geometry I   & Winter 2010 \\
    \end{tabular}
\end{itemize}

\noindent
AMATH 301 -- An undergraduate course in numerical
analysis. Computational solutions to linear systems, curve / data
fitting, numerical integration and differentiation, solutions to
differential equations, and optimization. \\

\noindent
AMATH 351 -- Standard techniques in solving first and second order
equations, series solutions, Laplace transform, systems of linear
equations and introductory linear analysis, systems of nonlinear
equations and perturbation theory.



\pagebreak



%% %%%%%%%%%%%%%%%%%%%%%%%%%%%%%%
%% \section*{References}
%% %%%%%%%%%%%%%%%%%%%%%%%%%%%%%%


%% \noindent
%% {\bf Bernard Deconinck}

%% Professor

%% University of Washington

%% Department of Applied Mathematics

%% Box 352420

%% Seattle, WA 98105-2420

%% 206.543.6069

%% {\tt bernard@amath.washington.edu} \\



%% \noindent
%% {\bf J. Nathan Kutz}

%% Professor and Chair

%% University of Washington

%% Department of Applied Mathematics

%% Box 352420

%% Seattle, WA 98105-2420

%% 206.685.3029

%% {\tt kutz@amath.washington.edu} \\



%% \noindent
%% {\bf William A. Stein}

%% Professor

%% University of Washington

%% Department of Mathematics

%% Box 354350

%% Seattle, WA 98195-4350

%% 206.543.1916

%% {\tt wstein@gmail.com}

\end{document}
