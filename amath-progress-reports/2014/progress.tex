\documentclass[11pt]{amsart}

\usepackage{fullpage}


\begin{document}
\begin{centering}
\large{\bf{PROGRESS REPORT: 2013-2014}}

\vspace{8pt}

CHRIS SWIERCZEWSKI

\vspace{8pt}

\today %15 April 2013

\vspace{8pt}

\end{centering}

\section*{Research Description}

My primary goal is to make Abelian functions and Riemann surfaces, two
mathematical objects seen in a very wide variety of applications, as
computationally accessible as trigonometric and hyperbolic
functions. Doing so would provide to the scientific community an
essential ingredient in finding large families of solutions to partial
differential equations arising in a variety of fields including plasma
physics, nonlinear optics, and water waves.

\section*{Research Update}

At the start of this academic year I completed the first major feature
of my software {\sc abelfunctions}: the calculation of period matrices
of Riemann surfaces. Period matrices are important not only in
applications to solutions to partial differential equations and
optimization but have a large theoretical interest as
well. Documentation for {\sc abelfunctions} can be found at {\tt
  http://abelfunctions.cswiercz.info}.

The next step was to implement the Abel map, a function mapping a
Riemann surface to its Jacobian. However, the initial implementation of
my algorithms lacked the infrastructure needed to write this algorithm
elegantly. I spent part of this year studying object-oriented
programming design patterns and software architecture design
\cite{Gamma1995, Martin2000}. Most of {\sc abelfunctions} is now
rewritten with these principles in place and the result has been not
only code that's easier to work with and read but better performing
code. All non-symbolic operations execute at least 80\% faster. This
sub-project is in line with my initial research goal of making an
easy-to-extend platform for computing with Riemann surfaces.

Grady Williams has completed his work on developing a high-performance
version of the Riemann theta function using GPUs. His implementation
provides at least a x100 speedup to the algorithm. A paper on further
improving Riemann theta performance via the Siegel transform is in the
works. Grady will be attending the Ph.D. program in Robotics at Georgia
Tech starting Autumn 2014.

I ended the winter quarter with the successful completion of my general
examination. My committee includes my advisor Bernard Deconinck as well
as Randy LeVeque and Robert O'Malley from the Applied Mathematics
department and William Stein and Rekha Thomas from Mathematics. The
major components of {\sc abelfunctions} that remain are the calculation
of the Riemann constant vector and building a framework for defining and
evaluating other types of differentials.  With the software development
portion of my work winding down I've begun a more extensive literature
review of the background and applications of these computational tools.

\section*{Expected Academic Schedule}

I expect to complete my degree at the end of Spring or Summer of 2015.


\bibliographystyle{amsplain}
\begin{thebibliography}{9}
\bibitem{Gamma1995} Gamma, Erich and Helm, Richard and Johnson, Ralph
  and Vlissides, John, \emph{Design Patterns: Elements of Reusable
    Object-oriented Software}, {Addison-Wesley Longman Publishing Co.,
    Inc.}, {Boston, MA, USA}, {1995}.

\bibitem{Martin2000} Martin, Robert \emph{Design Principles and Design
  Patterns}, \tt{www.objectmentor.com}, {2000}.


\end{thebibliography}


\end{document}

