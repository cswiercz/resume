\documentclass[11pt]{amsart}

\usepackage{fullpage}

\title{Progress Report: 2011-2012}
\author{Chris Swierczewski}
\date{20 April 2012}

\begin{document}
\begin{centering}
\large{\bf{PROGRESS REPORT: 2012-2013}}

\vspace{8pt}

CHRIS SWIERCZEWSKI

\vspace{8pt}

15 April 2013

\vspace{8pt}

\end{centering}

\section*{Research Description}

My primary goal is to lead the effort to provide the mathematical and
computational infrastructure that will transform the way we compute
with partial differential equations by making Abelian functions and
Riemann surfaces as computationally accessible as trigonometric and
hyperbolic functions thereby providing to the scientific community an
essential ingredient in finding large families of solutions to partial
differential equations arising in a variety of fields including plasma
physics, nonlinear optics, and water waves.

\section*{Research Update}

During this academic year I continued to add to {\sc abelfunctions}, a
Python library for computing with Abelian functions and Riemann
surfaces. ({\tt github.com/cswiercz/abelfunctions}) I used Deconinck
and van Hoeij's {\sc algcurves} package as reference but added
variations and improvements on algorithm components. The goal is to
have the functionality of {\sc algcurves} implemented in {\sc
  abelfunctions} by next Winter. Such an implementation will bring an
open-source, fully-documented computational Riemann surfaces software
package to the mathematics community. \\

I was an invited speaker at North Carolina State University's Symbolic
Computation seminar where I gave an extensive overview of {\sc
  abelfunctions} and theory behind the software. My discussions with
faculty members at NCSU led to incorporating Smale's $\alpha$-theory
into my code --- a technique for provably verifying the validity of
Newton's method. I also gave a conference talk based on selected
topics from the aforementioned lecture at the Eighth IMACS
International Conference on Nonlinear Evolution Equations and Wave
Phenomena for the ``Symbolic and numerical aspects of nonlinear
differential and difference equations'' organized session. \\

Grady Williams, an undergraduate in the Mathematics Department, joined
me during the summer in developing high-performance implementations of
Riemann theta function evaluation. We wrote optimized C and CUDA code
for computing Riemann theta on Graphics Processing Units (GPUs) --- a
specialized compute device for highly parallel calculations. We are
currently developing an algorithm for efficiently computing Riemann
theta in parallel over many choices of Riemann matrix by using the
Siegel Transform to take advantage of the problem's symplectic
geometry. If successful, we will submit a paper on this work. \\

This year I expanded my mathematical knowledge primarily by attending
reading courses and seminars. I initiated personal reading courses on
the topic of my research with my advisor, Bernard Deconinck. I also
attend weekly group readings on linear analysis and perturbation
theory; foundational topics within my general interests. I regularly
attend the Applied Mathematics Special Topics Seminar and the
Mathematical Methods Seminar as well as various other seminar sessions
within the Mathematics and Applied Mathematics departments.

\section*{Expected Academic Schedule}

I expect to complete my general examination by Winter 2014 and complete
my degree at the end of 2015.

\end{document}

