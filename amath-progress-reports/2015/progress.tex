\documentclass[12pt]{amsart}

\usepackage{fullpage}
\usepackage{footnote}

\begin{document}
\begin{centering}
\large{\bf{PROGRESS REPORT: 2014-2015}}

\vspace{8pt}

CHRIS SWIERCZEWSKI

%% \vspace{8pt}

%% \today

\vspace{24pt}

\end{centering}

\section*{Research Description}

My primary goal is to make Abelian functions and Riemann surfaces, two
mathematical objects seen in a very wide variety of applications, as
computationally accessible as trigonometric and hyperbolic functions. Doing so
would provide to the scientific community an essential ingredient in finding
large families of solutions to partial differential equations arising in a
variety of fields including plasma physics, nonlinear optics, and water waves.

\section*{Research Update}

My major contributions to my software package {\sc abelfunctions} \footnote{\tt
  http://abelfunctions.cswiercz.info} this year are the implementations of the
Abel map and the Riemann constant vector. The latter contribution to the
software is described in a paper {\it ``Computing the Riemann Constant
  Vector'',} co-authored with Bernard Deconinck and Matt Patterson. At the time
of writing this progress report the paper is submitted for publication in the
Journal for Symbolic Computation. One final ingredient for computing solutions
to the Kadomtsev--Petviashvili equation, a key partial differential equation
describing two-dimensional shallow water wave propagation, is needed and will
be available by the end of the academic year thus completing a major objective
of my research.

More researchers have been contacting me about the use of my software,
prompting some necessary refactoring of the code in order to improve
performance and readability as well as improved documentation. The primary
objective regarding these kinds of additions is to make it easier for future
students to contribute to the project. To help guide the development process I
have written an automated testing suite to help catch bugs and monitor the
stability of the code as changes are made. The suite is built using Python's
{\sc unittest} package and is executed automatically using the {\sc
  TravisCI} \footnote{\tt https://travis-ci.org} service.

My previous student, Grady Williams, graduated in Summer 2014 and is now a
Ph.D. student in the robotics department at Georgia Tech. I have taken on a
second undergraduate student, James Collins, who is assisting me in deriving
new techniques for computing the Riemann theta function; an essential special
function in the study of Riemann surfaces and Abelian functions. He has also
made contributions towards improving the performance of {\sc abelfunctions}.

\section*{Expected Academic Schedule}

I expect to complete my degree at the end of Autumn of 2015. Afterwards, I will
be staying for approximately two years as an Acting Assistant Professor.

\end{document}

