\documentclass[11pt]{amsart}

\usepackage{fullpage}

\title{Progress Report: 2011-2012}
\author{Chris Swierczewski}
\date{20 April 2012}

\begin{document}
\begin{centering}
\large{\bf{PROGRESS REPORT: 2011-2012}}

\vspace{12pt}

CHRIS SWIERCZEWSKI

\vspace{12pt}

14 MAY 2012

\vspace{24pt}
\end{centering}

\section*{Research Description}

My primary goal is to lead the effort to provide the mathematical and 
computational infrastructure that will transform the way we compute with 
partial differential equations by making Abelian functions as computationally 
accessible as trigonometric and hyperbolic functions thereby providing to the
scientific community an essential ingredient in finding large families of 
solutions to partial differential equations arising in a variety of fields 
including plasma physics, nonlinear optics, and water waves. 

\section*{Research Update}

I spent most of this academic year building the computational framework to my 
software package ``abelfunctions'', a Python-based library for computing with 
Abelian functions. The code is hosted at the online code repository website, 
Github, ({\tt https://www.github.com/cswiercz/abelfunctions}), where other 
researchers can freely browse and contribute to the source code. The decision 
to host the code here is in line with the proposals discussed at the
ICIAM 2011 satellite meeting on reproducible research as well as in support of 
the open-source software community.

I have also expanded my network of collaborators. Last academic year and during
the Autumn quarter of this academic year I presented my work at several 
conferences throughout the United States and at international conferences in 
Edinburgh, UK and Olso, Norway. From these conferences I've initiated 
collaborations with Harry Braden at the University of Edinburgh and Ond\u{r}ej 
Ce\u{r}t\'{i}k at the University of Nevada. This collaboration will lead to a more 
robust, fully featured software library.

\section*{Presentations and Coursework}

In October of this academic year I presented a talk titled ``Some 
Computational Problems Using Riemann Theta Functions in Sage'' at the AMS Fall
Western Sectional Meeting in Salt Lake City and at the SIAM Conference on 
Applied Algebraic Geometry in Chapel Hill. This talk became the basis of a
paper ``Computing Riemann theta functions in Sage with applications'', 
coauthored with Bernard Deconinck, that was submitted for publication at the 
end of 2011.

My coursework for this quarter includes a three-quarter algebraic geometry 
sequence offered by the Mathematics Department and a course on general purpose
GPU programming course offered by the Mechanical Engineering Department.
Algebraic geometry lies at the heart of some of the theory of Abelian 
functions. Although much is accomplished when considering only the ``classical''
theory, a background in modern algebraic geometry will allow me to take 
advantage of any useful tools in the field I may encounter. The course in 
general purpose GPU programming deeply enriched my computational skills and 
will allow me to develop much faster implementations of algorithms related 
to Abelian functions.

\section*{Expected Academic Schedule}

I expect to complete my general examination by the end of 2013 and complete
my degree by the end of 2015.

\end{document}

