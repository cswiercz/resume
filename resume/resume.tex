\documentclass{article}

\usepackage{fullpage}
\usepackage{mdwlist}
\usepackage{hyperref}
%\usepackage{enumitem}
%\usepackage{parskip}

%% \setlist{nolistsep}
%% \setlength{\parskip}{0pt}
%% \pagestyle{empty}

\begin{document}

\section*{Christopher J. Swierczewski \hspace{5.9cm} R\'esum\'e}

{\tt cswiercz@gmail.com} \\
253.223.3721

%%%%%%%%%%%%%%%%%%%%
\section*{Summary of Qualifications}
%%%%%%%%%%%%%%%%%%%%

\begin{itemize}
  \setlength{\itemsep}{0pt}
\item {\bf Mathematics}: linear analysis, numerical analysis, tensor methods,
  algebraic geometry, optimization, and machine learning.
\item {\bf Programming}: (Expert) Python, Cython. (Advanced) C, CUDA.
  (Intermediate) C++, OpenMP, MPI. (Additional) object-oriented design, software
  architecture, test-driven development.
\item Lead the design, development, and testing of high performance scientific
  software used to solve abstract mathematical problems using a huge range of
  mathematical and computational tools.
\item Dedicated to discovering optimal solution techniques and improving
  software performance by combining mathematical discovery and efficient
  computational methods.
\item Exceptional ability to clearly present technical information as evidenced
  by recognition for academic talks and teaching performance in computational
  mathematics courses.
\item Detailed in-person and remote collaborator with experience in
  pair-programming environment, code review processes, as well as in giving
  numerous talks and presentations.
\end{itemize}


%%%%%%%%%%%%%%%%%%%%%%%%%%%%
\section*{Professional Experience}
%%%%%%%%%%%%%%%%%%%%%%%%%%%%

\begin{itemize*}
  \setlength{\itemsep}{0pt}
  \item {\bf Applied Scientist}, Amazon Web Services - Amazon AI, Seattle, WA. May
    2017 - Present
    \begin{itemize*}
    \item Co-creator of SageMaker LDA, a topic modeling first-party algorithm
      provided by AWS SageMaker which is faster and more scalable than known
      open-source alternatives. Algorithm serves as key component to AWS
      Comprehend, an easy-to-use natural language processing product.
    \item Contributor to SageMaker RCF, a Random Cut Forest-based anomaly
      detection first-party algorithm provided by AWS SageMaker.
    \item Contributor to MXNet and Tensorly, a Python package for tensor learning.
    \end{itemize*}
  \item {\bf Software Development Engineer}, Amazon, Seattle, WA. November 2016
    - May 2017.
    \begin{itemize*}
    \item Expanded the space of advertising metrics which our
      Elasticsearch-based reporting backend can process and serve.
    \item Designed a serverless, automated, and generalized database auditing
      system using AWS Lambda and AWS DynamoDB.
    \end{itemize*}
  \item {\bf Research Mathematician}, Institute for Defense Analysis:
    Center for Communications Research, La Jolla, CA. June - August
    2012
    \begin{itemize*}
    \item Applied number theoretic, optimization, and statistical techniques to
      solving cryptographic problems in a high-performance computing setting.
    \end{itemize*}
  \item {\bf Software Developer}, Simulab Corporation, Seattle,
    WA. January 2009 - March 2009.
    \begin{itemize*}
    \item Applied Hidden Markov Models to problems in control theory and
      optimization with application to classifying surgical proficiency.
    \item Implemented mathematical and sensor data collection algorithms in C++.
    \end{itemize*}
  % \item {\bf Sage: Mathematics Software Developer}, Department of
  %   Mathematics, University of Washington, Seattle, WA. September 2007
  %   - September 2008.
  %   \begin{itemize*}
  %   \item Designed new Sage finance module around the Opentick financial data
  %     acquisition API. Devised methods of wrapping asynchronous functions in a
  %     synchonous environment.
  %   \item Designed tests and wrote documentation for mathematical functions in
  %     Python, Cython, and C/C++ under a UNIX environment.
  %   \end{itemize*}
  \item {\bf Applied Research Mathematician}, National Security Agency,
    Ft.Meade, MD. June - August 2007.
    \begin{itemize*}
    \item Applied algebraic, probabalistic, and statistical methods to improve
      cryptanalytic attacks against telecommunication encryption standards.
    \item Collaborated with mathematicians in researching cryptographic
      algorithm weaknesses. Implemented algorithms in C.
    \end{itemize*}
  % \item {\bf Teaching Assistant and Math Camp Counselor}, Department of
  %   Mathematics, University of Washington, Seattle, WA. June - August
  %   2005 and 2006.
\end{itemize*}


%%%%%%%%%%%%%%%%%%%%
\section*{Education}
%%%%%%%%%%%%%%%%%%%%


\begin{itemize}
  \setlength{\itemsep}{0pt}
  \item {\bf Ph.D. in Applied Mathematics}, University of Washington, Seattle,
    {\it In Progress}, \\ Thesis: {\it Computational Approach to Riemann Surfaces
    and the Kadomtsev-Petviashvili Equation} %\\ Advisor: Bernard Deconinck
    \begin{itemize*}
    \item Led the design and development of ``Abelfunctions'', an open-source
      mathematics software package for computing with Abelian functions and
      Riemann surfaces.
    \item Designed and improved new and existing algorithms using a wide variety
      of tools from various branches of mathematics, such as numerical analysis,
      linear analysis, and algebraic geometry; as well as using high-performance
      software development strategies.
    \item Used object-oriented design principles to create software that was
      easy for mathematicians to use yet maintained high performance standards
      while providing extendibility to future developers.
    \item Mentored team of bright undergraduate students in developing
      algorithms for quickly and accurately computing the Riemann theta
      function.
    \item Open-source code available on GitHub:
      \url{https://github.com/abelfunctions/abelfunctions}
    \end{itemize*}
  \item {\bf M.S. in Applied Mathematics}, University of Washington, Seattle,
    June 2010 \\ Thesis: {\it A Python Implementation of Chebyshev Functions}
    \begin{itemize*}
    \item Studied high-performance and high-accuracy function interpolation
      using Chebyshev polynomials.
    \item Created ``Pychebfun'', a Python library implementing these
      interpolation algorithms using tools from the Numpy and Scipy libraries:
      \url{https://github.com/cswiercz/pychebfun}
    \end{itemize*}
  \item {\bf B.S. in Mathematics with Distinction}, University of Washington,
    Seattle, June 2008 \\ Thesis: {\it Connections Between the Sato-Tate
      Conjecture and the Generalized Riemann
      Hyptothesis} % \\ Advisor: William Stein
   \begin{itemize}
   \item Proved equivalence of the Sato-Tate Conjecture and the Generalized
     Riemann Hypothesis for elliptic curves over the rational numbers.
   \item Performed computational experiments with elliptic $L$-functions to
     computationally verify the Sato-Tate conjecture and related number theory
     conjectures.
   \end{itemize}
\end{itemize}

% %%%%%%%%%%%%%%%%%%%%%%%%%%%%%%%%%%%%%%%
% \section*{Additional Research Projects}
% %%%%%%%%%%%%%%%%%%%%%%%%%%%%%%%%%%%%%%%

% \begin{itemize}
%   \setlength{\itemsep}{0pt}
%   \item {\bf Kaggle}
%     \begin{itemize}
%     \item Top 17\% in ``Digit Recognizer'': used a custom convolutional neural
%       network implementing batch learning and dropout for managing over-fitting.
%     \item Top 20\% in ``Titanic'': applied principal component analysis and
%       polynomial interpolation to data which is then fed into both decision tree
%       and neural network models.
%     \end{itemize}
%   \item {\bf Zipper}
%     \begin{itemize}
%     \item Advised graduate students on the development of ``Zipper'', a
%       high-performance library for computing with conformal maps.
%     \item Integrated the library into Sage and added a web-based, interactive
%       front-end. Open-source code available on Google Code:
%       \url{https://code.google.com/p/zipper}
%     \end{itemize}
% \end{itemize}


%%%%%%%%%%%%%%%%%%%%%%%%%%%%%%
\section*{Awards}
%%%%%%%%%%%%%%%%%%%%%%%%%%%%%%

\begin{itemize}
  \setlength{\itemsep}{0pt}
  \item Boeing Teaching Award, University of Washington, Applied
    Mathematics, June 2016.
  \item Student Chapter Award, Society for Industrial and Applied Mathematics
    (SIAM), June 2014.
  \item Boeing Service Award, University of Washington, Applied
    Mathematics, June 2013.
  \item American Mathetmatical Society Sectional Meeting Travel Grant,
    October 2011.
  \item University of Alaska Fairbanks Travel Grant, January 2011.
\end{itemize}
\end{document}



%% %%%%%%%%%%%%%%%%%%%%%%%%%%%%%%
%% \section*{References}
%% %%%%%%%%%%%%%%%%%%%%%%%%%%%%%%


%% \noindent
%% {\bf Bernard Deconinck}

%% Associate Professor

%% University of Washington

%% Department of Applied Mathematics

%% Box 352420

%% Seattle, WA 98105-2420

%% 206.543.6069

%% {\tt bernard@amath.washington.edu} \\



%% \noindent
%% {\bf J. Nathan Kutz}

%% Professor and Chair

%% University of Washington

%% Department of Applied Mathematics

%% Box 352420

%% Seattle, WA 98105-2420

%% 206.685.3029

%% {\tt kutz@amath.washington.edu} \\



%% \noindent
%% {\bf William A. Stein}

%% Professor

%% University of Washington

%% Department of Mathematics

%% Box 354350

%% Seattle, WA 98195-4350

%% 206.543.1916

%% {\tt wstein@gmail.com}

%% \end{document}
