\documentclass{article}

\usepackage{fullpage}
\usepackage{mdwlist}
\usepackage{hyperref}
%\usepackage{enumitem}
%\usepackage{parskip}

%% \setlist{nolistsep}
%% \setlength{\parskip}{0pt}
%% \pagestyle{empty}

\begin{document}

\section*{Christopher J. Swierczewski \hspace{5.9cm} R\'esum\'e}

%\begin{table}[h]%  \centering
\begin{tabular}{lrl}
  Department of Applied Mathematics \hspace{5cm} & Cell:    & 253.223.3721 \\
  University of Washington          & Fax:         & 206.685.1440            \\
  Lewis Hall \#304, Box 353925      & Website:     & \tt www.cswiercz.info   \\
  Seattle, WA 98195                 & E-mail:      & \tt cswiercz@gmail.com
\end{tabular}
%\end{table}

%%%%%%%%%%%%%%%%%%%%
\section*{Summary of Qualifications}
%%%%%%%%%%%%%%%%%%%%

\begin{itemize}
  \setlength{\itemsep}{0pt}
  \item {\bf Mathematics}: computational geometry, complex algebraic geometry,
    numerical analysis, computer algebra systems, and partial differential
    equations.
  \item {\bf Programming}: Python, Cython, C, C++, CUDA, Git, object-oriented
    design, software architecture, test-driven development, Matlab.
  \item Specialized in high performance symbolic-numerical software design and
    implementaion to solve problems in computational geometry, numerical
    analysis, and algebraic geometry.
  \item Experience with parallel environments through CUDA-based GPU
    development.
  \item Effective in-person and remote (via Git/Github) collaborator. Experience
    in pair-programming environment and with code review processeses.
  \item Dedicated to discovering optimal solutions techniques and improving
    software performance by focusing on mathematical and programmatical details.
  \item Cleared for Top Secret information and granted access to Sensitive
    Compartmented Information.
\end{itemize}


%%%%%%%%%%%%%%%%%%%%
\section*{Education}
%%%%%%%%%%%%%%%%%%%%


\begin{itemize}
  \setlength{\itemsep}{0pt}
  \item {\bf Ph.D. in Applied Mathematics}, University of Washington,
    Seattle, Expected March 2016 \\ Advisor: Bernard Deconinck
    \begin{itemize*}
    \item Designed and implementated algebraic and numerical tools for computing
      with Abelian functions and Riemann surfaces in an open-source mathematical
      software package, ``abelfunctions''.
    \item Applied research results to computing periodic solutions to a large
      class of nonlinear partial differential equations using techniques from
      computational geometry, numerical analysis, and algebraic geometry.
    \item Developed high performance code in both algebraic and numerical
      aspects of the software package using a Cython / C back-end with an easy
      to use Python front-end.
    \item Mentored junior team on quickly and accurately computing Riemann theta
      functions.
    \item Open-source code available on GitHub:
      \url{https://github.com/cswiercz/abelfunctions}
    \end{itemize*}
  \item {\bf M.S. in Applied Mathematics}, University of Washington,
    Seattle, June 2010 \\ Thesis: {\it A Python Implementation
      of Chebyshev Functions}
    \begin{itemize*}
    \item Studied high-performance and high-accuracy function interpolation
      using Chebyshev polynomials.
    \item Developed ``pychebfun'', an Python library implementing these
      interpolation algorithms using tools from the Numpy and Scipy libraries.
    \item Open-source code available on GitHub:
      \url{https://github.com/cswiercz/pychebfun}
    \end{itemize*}
  \item {\bf B.S. in Mathematics (Comprehensive) with Distinction},
    University of Washington, Seattle, June 2008 \\ Thesis: {\it
      Connections Between the Sato-Tate Conjecture and the Generalized
      Riemann Hyptothesis} \\ Advisor: William Stein
   \begin{itemize}
   \item Proved equivalence of the Sato-Tate Conjecture and the Generalized
     Riemann Hypothesis for elliptic curves over the rational numbers.
   \item Performed computational experiments with elliptic $L$-functions to
     computationally verify the Sato-Tate conjecture.
   \item Results published in American Mathematical Society Bulletin v.45 no.2
     (Fall 2007 - Spring 2008).
   \end{itemize}
\end{itemize}


%%%%%%%%%%%%%%%%%%%%%%%%%%%%
\section*{Professional Experience}
%%%%%%%%%%%%%%%%%%%%%%%%%%%%

\begin{itemize*}
  \setlength{\itemsep}{0pt}
  \item {\bf Research Mathematician}, Institute for Defense Analysis:
    Center for Communications Research, La Jolla, CA. June -- August
    2012
  \item {\bf Software Developer}, Simulab Corporation, Seattle,
    WA. January 2009 -- March 2009.
    \begin{itemize*}
    \item Researched theory and applications of Hidden Markov Models to problems
      in control theory and optimization.
    \item Implemented algorithms in a C++ back-end for EDGE, a device used in
      laproscopic surgion training and evaluation.
    \end{itemize*}
  \item {\bf Sage: Mathematics Software Developer}, Department of
    Mathematics, University of Washington, Seattle, WA. September 2007
    -- September 2008.
    \begin{itemize*}
      \item Designed new Sage finance module around the Opentick
        financial data acquisition API. Devised methods of wrapping
        asynchronous functions in a synchonous environment.
      \item Designed tests and wrote documentation for mathematical
        functions in Python, Cython, and C/C++ under a UNIX environment.
    \end{itemize*}
  \item {\bf Applied Research Mathematician}, National Security Agency,
    Ft. Meade, MD. June -- August 2007.
    \begin{itemize*}
      \item Applied algebraic, probabalistic, and statistical methods to
        improve cryptoanalytic attacks against telecommunication
        encryption standards.
      \item Collaborated with mathematicians in researching
        cryptographic algorithm weaknesses. Implemented algorithms in C.
    \end{itemize*}
  \item {\bf Teaching Assistant and Math Camp Counselor}, Department of
    Mathematics, University of Washington, Seattle, WA. June -- August
    2005 and 2006.
\end{itemize*}

%%%%%%%%%%%%%%%%%%%%%%%%%%%%%%%%%%%%%%%
\section*{Additional Research Projects}
%%%%%%%%%%%%%%%%%%%%%%%%%%%%%%%%%%%%%%%

\begin{itemize}
  \setlength{\itemsep}{0pt}
  \item {\bf Zipper Development}
    \begin{itemize}
    \item Advised graduate students on the development of ``Zipper'', a
      high-performance library for computing with conformal maps.
      \item Integrated the library into Sage and added a web-based, interactive
        front-end.
      \item Open-source code available on Google Code:
        \url{https://code.google.com/p/zipper}
    \end{itemize}
  \item {\bf CLAWPACK Development}
    \begin{itemize}
    \item Performed foundational work on conversion of CLAWPACK, a
      high-performance numerical partial differential equation solver, to a
      dynamic library.
    \item Attended Scipy 2009 conference on scientific computing in Python.
    \end{itemize}
\end{itemize}


% %%%%%%%%%%%%%%%%%%%%%%%%%%%%%%
% \section*{Teaching Experience}
% %%%%%%%%%%%%%%%%%%%%%%%%%%%%%%

% \begin{itemize}
%   \item {\bf Instructor}, University of Washington, Seattle:

%     \begin{tabular}{p{2.5cm}p{9.9cm}l}
%       - AMATH 301: & Beginning Scientific Computing      & Summer 2014 \\
%       - AMATH 301: & Beginning Scientific Computing      & Summer 2011 \\
%     \end{tabular}

%   \item {\bf Teaching Assistant}, University of Washington, Seattle:

%     \begin{tabular}{p{2.5cm}p{10cm}l}
%       - AMATH 301: & Beginning Scientific Computing      & Winter 2015 \\
%       - AMATH 351: & Introduction to Differential Equations and Applications & Autumn 2014 \\
%       - AMATH 301: & Beginning Scientific Computing      & Winter 2011 \\
%       - AMATH 301: & Beginning Scientific Computing      & Autumn 2010 \\
%       - MATH 125:  & Calculus with Analytic Geometry II  & Spring 2010 \\
%       - MATH 124:  & Calculus with Analytic Geometry I   & Winter 2010 \\
%     \end{tabular}
% \end{itemize}

% \noindent
% AMATH 301 -- An undergraduate course in numerical
% analysis. Computational solutions to linear systems, curve / data
% fitting, numerical integration and differentiation, solutions to
% differential equations, and optimization. \\

% \noindent
% AMATH 351 -- Standard techniques in solving first and second order
% equations, series solutions, Laplace transform, systems of linear
% equations and introductory linear analysis, systems of nonlinear
% equations and perturbation theory.



%% %%%%%%%%%%%%%%%%%%%%%%%%%%%%%%
%% \section*{Awards}
%% %%%%%%%%%%%%%%%%%%%%%%%%%%%%%%

%% \begin{itemize}
%%   \setlength{\itemsep}{0pt}
%%   \item Student Chapter Certificate of Recognition, Society for
%%     Industrial and Applied Mathematics (SIAM), June 2014.
%%   \item Boeing Service Award, University of Washington, Applied
%%     Mathematics, June 2013.
%%   \item American Mathetmatical Society Sectional Meeting Travel Grant,
%%     October 2011.
%%   \item University of Alaska Fairbanks Travel Grant, January 2011.
%% \end{itemize}


\end{document}



%% %%%%%%%%%%%%%%%%%%%%%%%%%%%%%%
%% \section*{References}
%% %%%%%%%%%%%%%%%%%%%%%%%%%%%%%%


%% \noindent
%% {\bf Bernard Deconinck}

%% Associate Professor

%% University of Washington

%% Department of Applied Mathematics

%% Box 352420

%% Seattle, WA 98105-2420

%% 206.543.6069

%% {\tt bernard@amath.washington.edu} \\



%% \noindent
%% {\bf J. Nathan Kutz}

%% Professor and Chair

%% University of Washington

%% Department of Applied Mathematics

%% Box 352420

%% Seattle, WA 98105-2420

%% 206.685.3029

%% {\tt kutz@amath.washington.edu} \\



%% \noindent
%% {\bf William A. Stein}

%% Professor

%% University of Washington

%% Department of Mathematics

%% Box 354350

%% Seattle, WA 98195-4350

%% 206.543.1916

%% {\tt wstein@gmail.com}

%% \end{document}
